\chapter{Running Text}

\section{Lorem Ipsum}

\lipsum[1-2]

\section{Modifiers}

The most important text modifiers are \emph{emphasis} and \texttt{teletype}.

\section{Glossary}

Use the \texttt{glossaries} packages for acronyms and alike.
Here is an example: first use \gls{dag} followed by second use \gls{dag}.

It is recommended to execute \texttt{\textbackslash{}glsresetall} when starting a new chapter.
There doesn't seem to be a sane way of automating this, you have to do that manually.
A code snippet could help.

\section{References}

This section explains various ways of include references.
Use footnotes for less important references.

\subsection{Hyperlinks}

Official university website: \url{https://www.uibk.ac.at/}

\subsection{Footnotes}

This text has a footnote\footnote{Footnote content goes here.} attached to it.

\subsection{Citations}

The book \emph{Principles of Program Analysis}~\cite{Nielson:ppa} is referenced here.

\subsection{Blockquotes}

Some text before a blockquote.

\begin{quote}
	\lipsum[3]
\end{quote}

Some text after a blockquote.

\section{Lists}

\subsection{Bullets}

\begin{itemize}
	\item One entry in the list
	\item Another entry in the list
	      \begin{itemize}
		      \item Nested list element
	      \end{itemize}
\end{itemize}

\subsection{Enumeration}

\begin{enumerate}
	\item One entry in the list
	\item Another entry in the list
	      \begin{enumerate}
		      \item Nested list element
	      \end{enumerate}
\end{enumerate}

\subsection{Description}

\begin{description}
	\item[First] One entry in the list
	\item[Second] Another entry in the list
\end{description}

Alternatively you may force line breaks after the descriptor.

\begin{description}
	\item[First]\hfill\\
	      One entry in the list
	\item[Second]\hfill\\
	      Another entry in the list
\end{description}

\section{Math}

\subsection{Inline Math and Display Math}

These are your basic math environments.
Inline math is recommended for short, concise expression, like $n = 42$ or $\mathcal{R} = \emptyset$.
Display math should be used for more complex stuff:

$$f(x) = \int_{-\infty}^\infty \hat f(\xi)\,e^{2 \pi i \xi x} \,d\xi$$

\subsection{Equations with Numbers}

This text refers to \cref{eq:some_equation}.

\begin{equation}\label{eq:some_equation}
	x^2 + y^2 = z^2
\end{equation}

\subsection{Theorem Environments}

\begin{definition}
	We simply define $x = 13$.
\end{definition}

\begin{theorem}
	Let's say $x$ is equal to $13$.
\end{theorem}

\begin{proof}
	Proof by definition.
\end{proof}

\subsection{Units}

Leverage the \texttt{siunitx} package to properly typeset numbers and units.
\num{12345678} and \SI{9.81}{\meter\per\second^2}.
