\documentclass[parskip=half]{scrbook}

\title{Title}
\subtitle{Subtitle}
\author{Firstname Lastname}
\date{1 January 2020}

\def\titlethesistype{Master Thesis}
\def\titlesupervisor{Supervisor Name}

\usepackage[utf8]{inputenc}

% Math
\usepackage{amsmath}
\usepackage{amsthm}
\newtheorem{definition}{Definition}[section]
\newtheorem{theorem}{Theorem}
\usepackage{amssymb}
\let\emptyset\varnothing

% SI Units
\usepackage[binary-units]{siunitx}
\sisetup{detect-all}

% References
\usepackage[numbers]{natbib}
\bibliographystyle{plainnat}

% Color
\usepackage{xcolor}
\definecolor{linkcolor}{cmyk}{1.0, 0.6, 0.0, 0.56}

% Links
\usepackage{hyperref}
\hypersetup{
	colorlinks = true,
	citecolor  = linkcolor,
	linkcolor  = linkcolor,
	urlcolor   = linkcolor,
}
\urlstyle{same}
\usepackage[noabbrev]{cleveref}

% Graphics
\usepackage{graphicx}
\usepackage{wrapfig}

% Title
\usepackage{titlepage}

% Page
\usepackage[margin=2cm,bottom=3cm,footskip=1cm]{geometry}
% \usepackage{showframe}
\usepackage{chngcntr}
\setcounter{tocdepth}{2}
\setcounter{secnumdepth}{2}
\counterwithout{equation}{chapter}
\counterwithout{figure}{chapter}
\counterwithout{table}{chapter}

% Font
% \usepackage{microtype}
\usepackage{newtxtext,newtxmath}
\usepackage[scale=0.88]{sourcecodepro}
\usepackage[T1]{fontenc}

% Custom packages
\usepackage{lipsum}

\begin{document}

\frontmatter
\maketitle
\tableofcontents

\chapter*{Abstract}

Abstract paragraph.

\mainmatter

\chapter{Running Text}

\section{Lorem Ipsum}

\lipsum[1-2]

\section{Modifiers}

The most important text modifiers are \emph{emphasis} and \texttt{teletype}.

\section{References}

This section explains various ways of include references.
Use footnotes for less important references.

\subsection{Hyperlinks}

Official university website: \url{https://www.uibk.ac.at/}

\subsection{Footnotes}

This text has a footnote\footnote{Footnote content goes here.} attached to it.

\subsection{Citations}

The book \emph{Principles of Program Analysis}~\cite{Nielson:ppa} is referenced here.

\subsection{Blockquotes}

Some text before a blockquote.

\begin{quote}
	\lipsum[3]
\end{quote}

Some text after a blockquote.

\section{Lists}

\subsection{Bullets}

\begin{itemize}
	\item One entry in the list
	\item Another entry in the list
	      \begin{itemize}
		      \item Nested list element
	      \end{itemize}
\end{itemize}

\subsection{Enumeration}

\begin{enumerate}
	\item One entry in the list
	\item Another entry in the list
	      \begin{enumerate}
		      \item Nested list element
	      \end{enumerate}
\end{enumerate}

\subsection{Description}

\begin{description}
	\item[First] One entry in the list
	\item[Second] Another entry in the list
\end{description}

Alternatively you may force line breaks after the descriptor.

\begin{description}
	\item[First]\hfill\\
	      One entry in the list
	\item[Second]\hfill\\
	      Another entry in the list
\end{description}

\section{Math}

\subsection{Inline Math and Display Math}

These are your basic math environments.
Inline math is recommended for short, concise expression, like $n = 42$ or $\mathcal{R} = \emptyset$.
Display math should be used for more complex stuff:

$$f(x) = \int_{-\infty}^\infty \hat f(\xi)\,e^{2 \pi i \xi x} \,d\xi$$

\subsection{Equations with Numbers}

This text refers to \cref{eq:some_equation}.

\begin{equation}\label{eq:some_equation}
	x^2 + y^2 = z^2
\end{equation}

\subsection{Theorem Environments}

\begin{definition}
	We simply define $x = 13$.
\end{definition}

\begin{theorem}
	Let's say $x$ is equal to $13$.
\end{theorem}

\begin{proof}
	Proof by definition.
\end{proof}

\subsection{Units}

Leverage the \texttt{siunitx} package to properly typeset numbers and units.
\num{12345678} and \SI{9.81}{\meter\per\second^2}.

\backmatter
\bibliography{references}

\end{document}
